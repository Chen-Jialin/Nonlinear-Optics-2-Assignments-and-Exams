\documentclass{assignment}
\ProjectInfos{非线性光学}{PHYS5252P}{2021-2022 学年第二学期}{作业二}{(随堂测验)}{陈稼霖}[https://github.com/Chen-Jialin]{SA21038052}

\begin{document}
\begin{prob}
    试比较体介质和光纤中非线性光学效应的区别.
\end{prob}
\begin{sol}
    
\end{sol}

\begin{prob}
    请解释光致折射率改变和光折变效应, 并简要说明二者异同.
\end{prob}
\begin{sol}
    
\end{sol}

\begin{prob}
    以三次谐波和受激拉曼散射为例, 比较参量过程和非参量过程的异同.
\end{prob}
\begin{sol}
    
\end{sol}

\begin{prob}
    分析材料中的色散与光纤中的色散有何共同点和不同点. 多模光纤和单模光纤中色散的主要来源是什么?
\end{prob}
\begin{sol}
    
\end{sol}

\begin{prob}
    考虑在光纤中输入一列幅度恒定为 $A_0=\sqrt{P_0}$ 的连续波, 若光纤在入射光波长上的色散为 $\beta_2$, 并忽略光纤损耗, 则光波在光纤中的传输方程为
    \[
        \frac{\partial\bar{A}}{\partial z}+\frac{i}{2}\beta_2\frac{\partial^2\bar{A}}{\partial T^2}=i\gamma\abs{\bar{A}}^2\bar{A}
    \]
    其中, $T=t-\beta_1z$. 证明: $\bar{A}=\sqrt{P_0}\exp(i\gamma P_0z)$ 是上述方程满足初始条件 $\bar{A}_0=\sqrt{P_0}$ 的解.
\end{prob}
\begin{pf}
    
\end{pf}

\begin{prob}
    目前光通信为什么采用以下三个波长: $\lambda_1=0.85\,\mu$m, $\lambda_2=1.31\,\mu$m, $\lambda_3=1.55\,\mu$m? 光纤通信为什么向长波场, 单模光纤方向发展?
\end{prob}
\begin{sol}
    
\end{sol}

\begin{prob}
    试由光波的传播效应证明产生光子回波的 (波矢) 相位匹配条件为 $k_3=2k_1-k_1$.
\end{prob}
\begin{pf}

\end{pf}

\begin{prob}
    一单模阶跃型折射率光纤, (1) 设 $a=5\,\mu$m, $n_2=1.5$, $\lambda=1\,\mu$m, 求次单模光纤可取的最大纤芯折射率; (2) 设 $n_1=1.501$, $n_2=1.5$, $\lambda=1\,\mu$m, 求此单模光纤可取的最大纤芯直径.
\end{prob}
\begin{sol}
    
\end{sol}

\begin{prob}
    设光入射通过一各向均匀介质, 考虑介质中的三阶非线性效应, 试证明:
    \begin{itemize}
        \item[(1)] 当入射光为左旋或右旋光时, 出射光仍为左旋或右旋光, 其折射率的变化为 $\delta n(circular)=\frac{A}{2\varepsilon_0n_0}\abs{E}^2$;
        \item[(2)] 当入射光为线偏振光时可以拆分为左旋和右旋光, 证明: 通过介质后的折射率变换为 $\delta n(linear)=\frac{(A+\frac{B}{2})}{2\varepsilon_0n_0}\abs{E}^2$;
        \item[(3)] 证明入射光为线偏光时出射光仍为线偏光.
    \end{itemize}
    注: $n_0$ 为不考虑非线性时介质的折射率, $A$ 和 $B$ 均为常数, $A=6\varepsilon_0\chi_{1122}$, $B=6\varepsilon_0\chi_{1221}$.
\end{prob}
\begin{pf}
    
\end{pf}
\end{document}