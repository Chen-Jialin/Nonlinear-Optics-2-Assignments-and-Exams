\documentclass{assignment}
\ProjectInfos{非线性光学}{PHYS5252P}{2021-2022 学年第二学期}{作业二}{(随堂测验)}{陈稼霖}[https://github.com/Chen-Jialin]{SA21038052}

\begin{document}
\begin{prob}
    试比较体介质和光纤中非线性光学效应的区别.
\end{prob}
\begin{sol}
    \begin{itemize}
        \item[(1)] 光波导中的光电场在一维或二维方向上被限制在一个波长量级的范围内传输, 波导中光功率密度大, 因而即使输入光功率较小, 仍可发生很强的非线性效应.
        \item[(2)] 波导中的光可以无衍射传播很长距离. 该距离仅为波导的不完善所引起的吸收或散射限制, 因而可提供足够长的介质实现强非线性效应.
        \item[(3)] 可以利用波导中导模的色散性质实现相位匹配.
        \item[(4)] 易于集成.
    \end{itemize}
\end{sol}

\begin{prob}
    请解释光致折射率改变和光折变效应, 并简要说明二者异同.
\end{prob}
\begin{sol}
    光致折射率改变: 非线性材料在光场照射下引起的材料折射率变化.\\
    光折变效应: 电光材料在光场照射下折射率随光强空间分布变化而变化.

    异:
    \begin{itemize}
        \item[(1)] 光折变效应本质为光场引发空间电荷场引起的线性电光效应, 为二阶非线性过程, 而光致折射率改变则为 SPM 和 XPM, 为三阶非线性效应; 光折变发生时间与光强相关 (光强越大, 则所需时间越短), 但光致折射率变化则几乎瞬时出现.
        \item[(2)] 光折变引起的光损伤可长时间保留, 也可擦除, 可用作信息存储; 光致折射率改变引起的透镜效应造成的光损伤是对材料的永久性破坏.
        \item[(3)] 光折变引起的光损伤可在低功率下发生 (无阈值), 最大光损伤的位置一般不在最大光强处 (非局域性); 光致折射率改变引起的光损伤仅当功率足够大 (大于损伤阈值) 时才会发生, 最大光损伤一定发生在光强最大处.
    \end{itemize}

    同:
    \begin{itemize}
        \item[(1)] 两种效应都会引起材料折射率的变化.
        \item[(2)] 都会破坏其他非线性效应, 有引起光损伤的可能性.
        \item[(3)] 都是在光场驱动下引起的.
    \end{itemize}
\end{sol}

\begin{prob}
    以三次谐波和受激拉曼散射为例, 比较参量过程和非参量过程的异同.
\end{prob}
\begin{sol}
    异:
    \begin{itemize}
        \item[(1)] 对介质: 参量过程前后介质状态不变, 介质不参与能量交换; 非参量过程前后介质状态改变, 介质能量交换, 引起介质内声子数量改变.
        \item[(2)] 对光场: 参量过程中能量守恒, 三次谐波湮灭 3 个频率为 $\omega_0$ 的基频光子, 产生一个频率为 $3\omega_0$ 的三倍频光子; 非参量过程中光子数守恒, 拉曼散射湮灭 1 个频率为 $\omega_p$ 的泵浦光子, 产生 1 个频率为 $\omega_s$ 的信号光子.
        \item[(3)] 相位匹配: 参量过程中, 需满足相位匹配条件, 光子的总动量守恒; 非参量过程中, 相位匹配条件在声子的辅助下自动满足, 光子的总动量不守恒.
    \end{itemize}

    同:
    \begin{itemize}
        \item[(1)] 都是三阶非线性过程.
        \item[(2)] 都借助了非线性介质中的虚能级.
    \end{itemize}
\end{sol}

\begin{prob}
    分析材料中的色散与光纤中的色散有何共同点和不同点. 多模光纤和单模光纤中色散的主要来源是什么?
\end{prob}
\begin{sol}
    体材料中的色散主要是材料自身特性引入的材料色散.

    光纤中色散包括:
    \begin{itemize}
        \item[(1)] 材料色散: 材料自身特性引入的色散.
        \item[(2)] 多模色散: 不同模式群速度不同.
        \item[(3)] 波导色散: 光纤中同一模式频率不同, 则传播常数不同.
        \item[(4)] 偏振模色散: 光纤不对称引起的偏振色散.
    \end{itemize}

    多模光纤中色散的主要来源是多模色散, 单模光纤中色散的主要来源是材料色散.
\end{sol}

\begin{prob}
    考虑在光纤中输入一列幅度恒定为 $A_0=\sqrt{P_0}$ 的连续波, 若光纤在入射光波长上的色散为 $\beta_2$, 并忽略光纤损耗, 则光波在光纤中的传输方程为
    \[
        \frac{\partial\bar{A}}{\partial z}+\frac{i}{2}\beta_2\frac{\partial^2\bar{A}}{\partial T^2}=i\gamma\abs{\bar{A}}^2\bar{A}
    \]
    其中, $T=t-\beta_1z$. 证明: $\bar{A}=\sqrt{P_0}\exp(i\gamma P_0z)$ 是上述方程满足初始条件 $\bar{A}_0=\sqrt{P_0}$ 的解.
\end{prob}
\begin{pf}
    先证明该函数满足初始条件: 当 $z=0$ 时, $\bar{A}=\sqrt{P_0}=A_0$.

    再证明该函数为传输方程的解: 将该函数代入传输方程得
    \begin{align}
        \text{方程左侧}=i\gamma P_0\sqrt{P_0}e^{i\gamma P_0}=\text{方程右侧}.
    \end{align}

    故 $\bar{A}=\sqrt{P_0}\exp(i\gamma P_0z)$ 是上述方程满足初始条件 $\bar{A}_0=\sqrt{P_0}$ 的解.
\end{pf}

\begin{prob}
    目前光通信为什么采用以下三个波长: $\lambda_1=0.85\,\mu$m, $\lambda_2=1.31\,\mu$m, $\lambda_3=1.55\,\mu$m? 光纤通信为什么向长波场, 单模光纤方向发展?
\end{prob}
\begin{sol}
    这三个波长在石英光纤中损耗、色散较小, 适合长距离高速通信.

    \begin{itemize}
        \item[(1)] 早期由于光源和探测器响应波长的限制, 将损耗较小的 $\lambda_1=0.85\,\mu$m 作为工作波长.
        \item[(2)] $\lambda_2=1.31\,\mu$m 为石英光纤的零色散点, 且损耗较小.
        \item[(3)] $\lambda_3=1.55\,\mu$m 为石英光纤损耗的极小点.
    \end{itemize}

    长波场、单模光纤有较好的传输特性:
    \begin{itemize}
        \item[(1)] 损耗: 波长越长, 光纤损耗越小, 适合信号长距离传输.
        \item[(2)] 色散: 单模光纤无多模色散, 适合长距离、高速的信号传输.
    \end{itemize}
\end{sol}

\begin{prob}
    试由光波的传播效应证明产生光子回波的 (波矢) 相位匹配条件为 $k_3=2k_1-k_1$.
\end{prob}
\begin{pf}
    设探测器相对原子位矢为 $\bm{l}$, 探测到第一、第二个脉冲的起止时刻分别为 $t_1$, $t_1'$, $t_2$, $t_2'$, 光子回波发生于 $t_3$ 时刻.
    光子回波现象中, 脉冲时间间隔满足
    \begin{gather}
        \left(t_3-\frac{\bm{k}_3\cdot\bm{l}}{\omega}\right)-\left(t_2'-\frac{\bm{k}_2\cdot\bm{l}}{\omega}\right)=\left(t_2-\frac{\bm{k}_2\cdot\bm{l}}{\omega}\right)-\left(t_1-\frac{\bm{k}_1\cdot\bm{l}}{\omega}\right)\\
        \Longrightarrow(\bm{k}_3-2\bm{k}_2+\bm{k}_1)\cdot\bm{l}=\omega(t_3-t_2'-t_2+t_1')=0.
    \end{gather}
    由于上式对任意 $\bm{l}$ 均成立, 故
    \begin{gather}
        t_3-t_2'=t_2-t_1',\\
        \bm{k}_3=2\bm{k}_2-\bm{k}_1.
    \end{gather}
\end{pf}

\begin{prob}
    一单模阶跃型折射率光纤, (1) 设 $a=5\,\mu$m, $n_2=1.5$, $\lambda=1\,\mu$m, 求次单模光纤可取的最大纤芯折射率; (2) 设 $n_1=1.501$, $n_2=1.5$, $\lambda=1\,\mu$m, 求此单模光纤可取的最大纤芯直径.
\end{prob}
\begin{sol}
    归一化频率
    \begin{align}
        f=k_0a(n_1^2-n_2^2)^{1/2}=\frac{2\pi}{\lambda}a(n_1^2-n_2^2)^{1/2}.
    \end{align}
    \begin{itemize}
        \item[(1)] 要使光纤仅支持单模, 则
        \begin{align}
            f\leq 2.415,
        \end{align}
        即最大线性折射率 $n_1=1.502$.
        \item[(2)] 要使光纤仅支持单模, 则
        \begin{align}
            f\leq 2.415,
        \end{align}
        即最大纤芯直径为 $7.016\,\mu$m.
    \end{itemize}
\end{sol}

\begin{prob}
    设光入射通过一各向均匀介质, 考虑介质中的三阶非线性效应, 试证明:
    \begin{itemize}
        \item[(1)] 当入射光为左旋或右旋光时, 出射光仍为左旋或右旋光, 其折射率的变化为 $\delta n(circular)=\frac{A}{2\varepsilon_0n_0}\abs{E}^2$;
        \item[(2)] 当入射光为线偏振光时可以拆分为左旋和右旋光, 证明: 通过介质后的折射率变换为 $\delta n(linear)=\frac{(A+\frac{B}{2})}{2\varepsilon_0n_0}\abs{E}^2$;
        \item[(3)] 证明入射光为线偏光时出射光仍为线偏光.
    \end{itemize}
    注: $n_0$ 为不考虑非线性时介质的折射率, $A$ 和 $B$ 均为常数, $A=6\varepsilon_0\chi_{1122}$, $B=6\varepsilon_0\chi_{1221}$.
\end{prob}
\begin{pf}
    各向同性介质仅有 3 个独立的三阶非线性张量元 $\chi_{1122}^{(3)}$, $\chi_{1212}^{(3)}$, $\chi_{1221}^{(3)}$, $\chi_{1111}^{(3)}=\chi_{1122}^{(3)}+\chi_{1212}^{(3)}+\chi_{1221}^{(3)}$, 且对于 $\omega=\omega+\omega-\omega$ 的三阶非线性过程, 由本征对称性, $\chi_{1122}^{(3)}=\chi_{1212}^{(3)}$, 故 $\chi_{1111}^{(3)}=6\chi_{1122}^{(3)}+\chi_{1221}^{(3)}$.
    折射率变化为 $\delta n=\sqrt{1+\chi_L+\chi_{NL}}-\sqrt{1+\chi_L}=\frac{2\chi_{NL}}{n_0}$.

    先证 (2)(3): 不妨选线偏振方向为 $x$, 则 $\bm{E}=\abs{E}e^{-i(\omega t-kz)}\hat{x}$.
    该非线性过程的简并因子 $D=\frac{3!}{2!}=3$.
    极化强度 $P_{NL}^{(3)}=3\varepsilon_0\chi^{(3)}\bm{E}(\omega)\bm{E}(\omega)\bm{E}(-\omega)=3\varepsilon_0\abs{E}^3\chi^{(3)}(\omega,\omega,-\omega)\hat{x}\hat{x}\hat{x}e^{-i(\omega t-kz)}=3\varepsilon_0\abs{E}^3\chi_{1111}^{(3)}e^{-i(\omega t-kz)}\hat{x}=3\varepsilon_0\chi_{1111}^{(3)}\abs{E}^2\bm{E}$, 故 $\chi_{LN}=3\chi_{1111}^{(3)}\abs{E}^2$, $\delta n(linear)=\frac{6\chi_{1122}^{(3)}+3\chi_{1221}^{(3)}}{2n_0}\frac{\varepsilon_0}{\varepsilon_0}\abs{E}^2=\frac{(A+\frac{B}{2})}{2n_0\varepsilon_0}\abs{E}^2$.

    再证 (1), 左/右旋圆偏光为 $\bm{E}=\frac{\abs{E}}{\sqrt{2}}(\hat{x}+\hat{y}e^{\pm\frac{\pi}{2}})e^{-i(\omega t-kz)}$, 琼斯矩阵为 $\bm{E}=\frac{\abs{E}}{\sqrt{2}}\begin{bmatrix}
        1\\
        \pm i
    \end{bmatrix}e^{-i(\omega t-kz)}$, 故 $\bm{E}\bm{E}\bm{E}^*=\frac{1}{2\sqrt{2}}\abs{E}^3e^{-i(\omega t+kz)}(\hat{x}\hat{x}\hat{x}\mp i\hat{x}\hat{x}\hat{y}\pm i\hat{x}\hat{y}\hat{y}+\hat{x}\hat{y}\hat{y}\pm i\hat{y}\hat{x}\hat{x}+\hat{y}\hat{x}\hat{y}-\hat{y}\hat{y}\hat{x}\pm i\hat{y}\hat{y}\hat{y})$.
    在 $x$ 方向, $P_{NL}^{(3)}=3\varepsilon_0[\chi_{1111}^{(3)}+\chi_{1122}^{(3)}+\chi_{1212}^{(3)}-\chi_{1221}^{(3)}]\frac{1}{2\sqrt{2}}\abs{E}^3e^{-i(\omega t-kz)}=6\varepsilon_0\chi_{1122}^{(3)}\abs{E}^2E_x$;
    同理在 $y$ 方向, $P_{NL}^{(3)}=\pm 3\varepsilon_0[\chi_{2222}^{(3)}+\chi_{2121}^{(3)}+\chi_{2211}^{(3)}-\chi_{2112}^{(3)}]\frac{1}{2\sqrt{2}}\abs{E}^3e^{-i(\omega t-kz)}=6\varepsilon_0\chi_{1122}^{(3)}\abs{E}^2E_y$.
    故 $\bm{P}_{NL}^{(3)}=P_{NL,x}^{(3)}\hat{x}+P_{NL,y}^{(3)}\hat{y}=6\varepsilon_0\chi_{1122}^{(3)}\abs{E}^2(E_x\hat{x}+E_y\hat{y})=6\varepsilon_0\chi_{1122}^{(3)}\abs{E}^2\bm{E}$.
    同理, 非线性极化强度仍正比于入射光, 因此不产生其他偏振分量, 仍为左/右旋圆偏光, $\delta n(circular)=\frac{6\varepsilon_0\varphi_{1122}^{(3)}}{2n_0\varepsilon_0}\abs{E}^2=\frac{A}{2n_0\varepsilon_0}\abs{E}^2$.
\end{pf}
\end{document}