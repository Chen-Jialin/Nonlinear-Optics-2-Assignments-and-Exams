\documentclass{assignment}
\ProjectInfos{非线性光学}{PHYS5252P}{2021-2022 学年第二学期}{作业一}{(随堂测验)}{陈稼霖}[https://github.com/Chen-Jialin]{SA21038052}

\begin{document}
\begin{prob}
    利用 KDP 晶体进行参量放大, 若其中有两个光波是 e 光, 第三个光波是 o 光, 试推导相位匹配角公式, 其中信号、空闲和泵浦光中哪一个为寻常光? 令 $\omega_3=$\SI{10000}{cm^{-1}}, $\omega_1=\omega_2=$\SI{5000}{cm^{-1}} 能否实现这种形式的相位匹配?
\end{prob}
\begin{sol}
    
\end{sol}

\begin{prob}
    试证明外加直流电场 $E_x=E_{0j}$ 的 KDP 晶体, 光波在 $zox$ 面内、与 $x$ 轴成 $45^{\circ}$ 方向传播时的电光延迟为下式. 其中 $l$ 为沿光传播方向上的晶体长度, $d$ 为沿外加电场的晶体厚度, $U_y$ 为外加电压.
    \[
        \Delta\varphi\approx\frac{2\pi l}{\lambda}\left[\frac{\sqrt{2}n_on_e}{\sqrt{n_o^2+n_e^2}}-n_o+\sqrt{2}\left[\frac{1}{n_o^2}+\frac{1}{n_e^2}\right]^{-2/3}\gamma_{41}U_y\frac{l}{d}\right].
    \]
\end{prob}
\begin{pf}

\end{pf}

\begin{prob}
    考虑 \ce{LiNbO_3} 浸提中的 II 型 ($o+e\rightarrow e$) 相位匹配下的共线传播倍频过程 $\omega+\omega\rightarrow 2\omega$:
    \begin{itemize}
        \item[1)] 设光波矢沿 $(\theta,\varphi)$, 求出此时有效非线性系数 $d_{\text{eff}}$ 的表达式.
        注: 已知 \ce{LiNbO_3} 晶体 (负单轴晶体) 的非线性系数矩阵为
        \[
            \begin{Bmatrix}
                0&0&0&0&d_{15}&-d_{22}\\
                -d_{22}&d_{22}&0&d_{15}&0&0\\
                d_{31}&d_{31}&d_{33}&0&0&0
            \end{Bmatrix}.
        \]
        \item[2)] 用折射率曲面的方法画出实现相位匹配的示意图.
        \item[3)] 若要得到最佳倍频输出, 问光波矢的方向 $(\theta,\varphi)$ 应取何值?
    \end{itemize}
\end{prob}
\begin{sol}
    \begin{itemize}
        \item[1)] 
        \item[2)] 
        \item[3)] 
    \end{itemize}
\end{sol}

\begin{prob}
    试证明在非共线相位匹配的条件下, 为获得远红外差频光 ($\omega_1$、$\omega_2\gg\omega_3$), 晶体必须具有反常色散特性.
\end{prob}
\begin{pf}
    
\end{pf}

\begin{prob}
    试证明, 如果二次谐波产生过程的基频光 $\omega$ 是寻常光, 倍频光 $2\omega$ 是非常光, $\theta_m$ 是其相位匹配角, 则有
    \[
        \Delta k(\theta)L|_{\theta=\theta_m}=\frac{2\omega L}{c}\sin(2\theta_m)\frac{(n_e^{2\omega})^{-2}-(n_o^{2\omega})^{-2}}{2(2n_o^{2\omega})^{-3}}(\theta-\theta_m).
    \]
\end{prob}
\begin{pf}
    
\end{pf}

\begin{prob}
    参量振荡器可以看作是一种新型的激光器, 利用它可以实现频率的调谐输出, 请以负单轴 II 类晶体为例, 分析角度调谐输出.
\end{prob}
\begin{sol}
    
\end{sol}

\begin{prob}
    请以负单轴晶体为例, 按以下条件分别推导 I 型和 II 型二次谐波的匹配带宽:
    \begin{itemize}
        \item[1)] 简并共线;
        \item[2)] 简并非贡献,
    \end{itemize}
    并比较 I 型和 II 型的带宽特点.
\end{prob}
\begin{pf}
    \begin{itemize}
        \item[1)] 
        \item[2)] 
    \end{itemize}
\end{pf}
\end{document}